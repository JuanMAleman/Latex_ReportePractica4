\subsection{Conclusión Campos Flores Santiago}

The bipolar transistor (BJT) is one of the fundamental devices in analog electronics, standing out for its versatility as an amplifier, switch, and control element. Its operation is based on the control of currents through the injection of minority carriers, allowing signal amplification with reasonable linearity in the active region and fast switching in saturation/cutoff.
Su sensibilidad térmica y menor eficiencia respecto a tecnologías modernas limitan su uso en aplicaciones de ultra bajo consumo, pero su principio de operación sigue siendo fundamental para comprender la electrónica de semiconductores. El dominio del BJT proporciona bases sólidas para el diseño de circuitos analógicos avanzados y la transición a tecnologías más modernas.