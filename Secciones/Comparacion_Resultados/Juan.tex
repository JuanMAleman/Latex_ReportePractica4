\subsection{Common-Emitter Amplifier with Fixed Bias Circuit}
The results for the Q-point operation, as well as the measurements of the output Vpp and input Vpp, 
differ significantly between the practical measurements performed in the laboratory, the theoretical 
calculations, and the simulated measurements. This is mainly because the laboratory measurements are not 
ideal; there are multiple varying factors that cause the practical measurements to deviate the most. The 
simulated measurements and theoretical calculations show somewhat closer agreement. In this case, the data 
may deviate because the values used in calculations are rounded, and it is also possible that certain 
factors are not considered when calculating the values, leading to differences.

\subsection{Common-Emitter Amplifier with Emitter-Stabilized Bias Circuit}
The results from the practical measurements, the theoretical calculations, and the simulated measurements 
in this case were more similar than in the previous circuit, likely because the conditions in all three 
cases had a higher degree of similarity. Unlike the previous circuit, we can observe that adding RE 
results in a lower gain, which allows us to infer that changes involving an increase in the resistive 
component of the circuit will affect the amount of gain obtained from the circuit.