\newpage
\section{Questionnaire}
\begin{enumerate}
    \item What does the resistor Rg represent? R = The resistor Rg represents the signal source resistance or the input resistance that limits the current entering the base of the transistor and helps stabilize the input signal.
    \item Why is the output signal inverted with respect to the input signal in the previous circuits? R = The output signal is inverted because, in a common-emitter amplifier, when the input voltage increases, the transistor conducts more, causing the collector voltage to decrease. This results in a 180-degree phase shift between the input and output signals.
    \item Which circuit provides the highest voltage gain? R = The fixed-bias amplifier without emitter resistance (RE) provides the highest voltage gain, since there is no emitter feedback to reduce the gain.
    \item Describe the effect of placing the 10 $\mu$F capacitor in the last three circuits. R = The 10 $\mu$F capacitor acts as a bypass capacitor that short-circuits the emitter resistor for AC signals, increasing the amplifier's gain by reducing negative feedback for AC while keeping DC bias conditions stable.
    \item Why are the capacitors C1 and C2 placed in all the circuits? R = The capacitors C1 and C2 are coupling capacitors. They block DC components while allowing AC signals to pass through, preventing DC bias from one stage from affecting the next stage or the signal source.
\end{enumerate}