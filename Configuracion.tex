% Configuración
\usepackage[utf8]{inputenc}
\usepackage[T1]{fontenc}
\usepackage{babel}

% Paquetes básicos y de utilidades
\usepackage{abstract}
\usepackage{amsmath}
\usepackage{caption}
\usepackage{chngcntr}
\usepackage{csquotes}
\usepackage{enumitem}
\usepackage{fancyhdr}
\usepackage{fix-cm}
\usepackage{float}
\usepackage{geometry}
\usepackage{graphicx}
\usepackage{listings}
\usepackage{parskip}
\usepackage{pgfplots}
\usepackage{subcaption}
\usepackage{tikz}
\usepackage{titlesec}
\usepackage{tocloft}
\usepackage{url}
\usepackage{xcolor}
\usepackage{etoolbox}
\usepackage{newtxtext}
\usepackage{newtxmath}
\usepackage{booktabs}
\usepackage{siunitx}

% Configuración de páginas y espaciado
\geometry{margin=1in}
\setlength{\absleftindent}{0pt}
\setlength{\absrightindent}{0pt}
\setlength{\headheight}{15.4pt}
\setlength{\parskip}{1em}
\pagestyle{plain}

% Configuración de tikz
\usetikzlibrary{trees, positioning, arrows.meta}
\pgfplotsset{compat=1.18}

% Configuración de URLs
\apptocmd{\UrlBreaks}{\do\/\do-}{}{}
% \setcounter{biburlnumshorthandthreshold}{100}
% \setcounter{biburlucshorthandthreshold}{100}
% \setcounter{biburllcshorthandthreshold}{100}

% HYPERREF DEBE IR ANTES DE BIBLATEX PERO DESPUÉS DE CASI TODO
\usepackage[breaklinks=true]{hyperref}
\usepackage{xurl} % Debe ir después de hyperref

% Configuración de hyperref
\hypersetup{
    colorlinks=true,
    linkcolor=black,
    filecolor=magenta,
    urlcolor=cyan,
    pdftitle={Practica No4},
    pdfauthor={Juan Manuel Manriquez Aleman},
    bookmarksopen=true,
    linktoc=all
}

% BIBLATEX debe ir después de hyperref
\usepackage[backend=biber, style=ieee, sorting=none]{biblatex}

% Configuración de bibliografía
\addbibresource{Bibliografia.bib}
\defbibheading{bibliography}[\refname]{
    \section*{#1}
    \markboth{#1}{#1}
}

% Configuración de títulos
\titleformat{\section}
  {\normalfont\fontsize{16}{19}\bfseries\selectfont}
  {\thesection}
  {1em}
  {}
\titleformat{\subsection}
  {\normalfont\fontsize{14}{17}\bfseries\selectfont}
  {\thesubsection}
  {1em}
  {}
\titleformat{\subsubsection}
  {\normalfont\fontsize{12}{14}\bfseries\selectfont}
  {\thesubsubsection}
  {1em}
  {}

\titlespacing*{\section}{0pt}{3.5ex plus 1ex minus .2ex}{2.3ex plus .2ex}
\titlespacing*{\subsection}{0pt}{3.25ex plus 1ex minus .2ex}{1.5ex plus .2ex}
\titlespacing*{\subsubsection}{0pt}{3.25ex plus 1ex minus .2ex}{1.5ex plus .2ex}

% Configuración de encabezados
\pagestyle{fancy}
\fancyhf{}
\fancyhead[L]{\leftmark}
\fancyfoot[C]{\thepage}
\renewcommand{\headrulewidth}{0.4pt}
\renewcommand{\footrulewidth}{0pt}

% Configuración de índices
\renewcommand{\cftsecleader}{\cftdotfill{\cftdotsep}}
\addto\captionsspanish{
  \renewcommand{\listtablename}{Índice de tablas}
  \renewcommand{\tablename}{Tabla}
  \renewcommand{\figurename}{Imagen}
}
\renewcommand{\lstlistlistingname}{Índice de códigos}

% Configuración de captions
\captionsetup[figure]{name=Figure}

% Configuración de anexos
\newcounter{anexo}
\renewcommand{\theanexo}{Anexo~\arabic{anexo}}

\newenvironment{anexo}[1][]{
  \refstepcounter{anexo}
  \par\noindent\textbf{\theanexo. #1}\par
}{\par}

% Configuración de nombres especiales
\renewcommand{\abstractname}{\fontsize{12}{14}\selectfont Resumen}
\renewcommand{\lstlistlistingname}{\fontsize{12}{14}\selectfont Índice de códigos}
\renewcommand{\listtablename}{\fontsize{12}{14}\selectfont Índice de tablas}

% Configuraciones de listados (mantener igual)
\lstdefinestyle{codigoPython}{
    language=Python,
    basicstyle=\ttfamily\footnotesize,
    keywordstyle=\color{blue}\bfseries,
    commentstyle=\color{green!50!black}\itshape,
    stringstyle=\color{red},
    numbers=left,
    numberstyle=\tiny\color{gray},
    stepnumber=1,
    numbersep=5pt,
    backgroundcolor=\color{white},
    showspaces=false,
    showstringspaces=false,
    showtabs=false,
    frame=single,
    rulecolor=\color{black},
    tabsize=4,
    captionpos=b,
    breaklines=true,
    breakatwhitespace=true,
    morekeywords={np, plt, subprocess, os, struct},
    literate= % Para caracteres especiales
        {á}{{\'a}}1 {é}{{\'e}}1 {í}{{\'i}}1 {ó}{{\'o}}1 {ú}{{\'u}}1
        {Á}{{\'A}}1 {É}{{\'E}}1 {Í}{{\'I}}1 {Ó}{{\'O}}1 {Ú}{{\'U}}1
        {ñ}{{\~n}}1 {Ñ}{{\~N}}1 {¿}{{?`}}1 {¡}{{!`}}1
}

% Configuración para código C
\lstdefinestyle{codigoC}{
    language=C,
    basicstyle=\ttfamily\footnotesize,
    keywordstyle=\color{blue}\bfseries,
    commentstyle=\color{green!50!black}\itshape,
    stringstyle=\color{red},
    numbers=left,
    numberstyle=\tiny\color{gray},
    stepnumber=1,
    numbersep=5pt,
    backgroundcolor=\color{white},
    showspaces=false,
    showstringspaces=false,
    showtabs=false,
    frame=single,
    rulecolor=\color{black},
    tabsize=4,
    captionpos=b,
    breaklines=true,
    breakatwhitespace=true,
    morekeywords={size_t, bool, true, false, pragma, omp},
    literate= % Para caracteres especiales
        {á}{{\'a}}1 {é}{{\'e}}1 {í}{{\'i}}1 {ó}{{\'o}}1 {ú}{{\'u}}1
        {Á}{{\'A}}1 {É}{{\'E}}1 {Í}{{\'I}}1 {Ó}{{\'O}}1 {Ú}{{\'U}}1
        {ñ}{{\~n}}1 {Ñ}{{\~N}}1 {¿}{{?`}}1 {¡}{{!`}}1}

% Estilo para terminal/consola
\lstdefinestyle{terminalstyle}{
    language=bash,
    basicstyle=\ttfamily\footnotesize\color{white},
    backgroundcolor=\color{black!90},
    numbers=none,
    frame=tb,
    framerule=0pt,
    framesep=3pt,
    rulecolor=\color{gray},
    xleftmargin=10pt,
    xrightmargin=10pt,
    breaklines=true,
    breakatwhitespace=true,
    showstringspaces=false,
    tabsize=2,
    % Colores para elementos específicos
    keywordstyle=\color{cyan}, % Comandos
    commentstyle=\color{green!60}, % Comentarios
    stringstyle=\color{yellow}, % Rutas y strings
    morekeywords={sudo, apt, git, python, gcc, make, cd, ls, echo}, % Comandos comunes
    morecomment=[l]{\#}, % Comentarios con #
    morestring=[b]", % Strings entre comillas
    literate= % Caracteres especiales
        {á}{{\'a}}1 {é}{{\'e}}1 {í}{{\'i}}1 {ó}{{\'o}}1 {ú}{{\'u}}1
        {Á}{{\'A}}1 {É}{{\'E}}1 {Í}{{\'I}}1 {Ó}{{\'O}}1 {Ú}{{\'U}}1
        {ñ}{{\~n}}1 {Ñ}{{\~N}}1 {¿}{{?`}}1 {¡}{{!`}}1
        {\$}{{\textcolor{green}{\$}}}1 % Prompt de terminal
        {>}{{\textcolor{green}{>}}}1   % Prompt secundario
        {*}{{\textcolor{red}{*}}}1     % Wildcards
        {~}{{\textcolor{blue}{\textasciitilde}}}1 % Directorio home
}

% Configuracion General Codigos
\renewcommand{\lstlistingname}{Código}

% Configuración para evitar underfull hboxes
\hbadness=10000